\begin{proof}
    A \(n\)-dimensional simplex has \(n+1\) vertices in \(\R^n\): \(V = \{v_0, v_1, \dots, v_n\}\). Specifically, the \(n\)-dimensional simplex is the convex hull \(C(V)\) of the set of the vertices. Moreover, I know that a \(p\)-face of the \(n\)-dimensional simplex is just a lower dimensional simplex with \(p+1\) vertices.\par
    How many different subsets of \(p+1\) distinct elements of \(V\) can I find? The number of combinations is equal to the binomial coefficient \(\binom{n+1}{p+1}\). But am I sure that the convex hulls related to these subsets are all distinct? That is, am I sure that I have exactly \(\binom{n+1}{p+1}\) distinct \(p\)-faces?
    Let \(V_1\) and \(V_2\) be two different subsets of \(p+1\) elements of \(V\) (\(V_1 \subset V, V_2 \subset V, V_1 \ne V_2\)). Let \(v \in V_1, v \notin V_2\). Suppose that \(v \in C(V_2)\). It would be true that \(v\) can be written as convex combination of the points in \(V_2\). This implies that \(v\) could be also written as convex combination of the points in \(V\). But this is a contradiction, since \(v\) is a vertex of the simplex (an  extremal point of the simplex can't be written as a convex combination of the vertices). Thus, \(v \notin C(V_2)\) and \(C(V_1) \neq C(V_2)\). I can conclude that a \(n\)-dimensional simplex has exactly \(\binom{n+1}{p+1}\) distinct \(p\)-faces: you just have to select every possible subset of \(p+1\) elements from the set of the vertices, and keep the related convex hulls.
\end{proof}