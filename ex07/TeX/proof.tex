\begin{proof}
    Instead of solving the problem
    \begin{align*}
        \text{maximize } & y^Tx \\
        \text{subject to } & x^TQx\leq 1
    \end{align*}
    I can solve the related problem
    \begin{align*}
        \text{minimize } & -y^Tx \\
        \text{subject to } & x^TQx\leq 1
    \end{align*}
    First I can write the Lagrangian function for the problem (the sum of the objective function and the weighted constraints):
    \[L(x,\mu) = -yx^T + \mu(x^TQx - 1)\]
    Then, the Kuhn-Tucker conditions become:
    \begin{gather}
        -y^T + 2\mu x^TQ = 0 \label{KKT1} \\
        \mu(x^TQx - 1) = 0 \label{KKT2}
    \end{gather}
    In order to find the points that satisfy these conditions, I have to explore various cases. Specifically, \eqref{KKT2} must hold. It follows that:
    \[\mu = 0 \lor x^TQx = 1\]
    \begin{enumerate}
        \item \(\mu = 0\)\par
        Since \eqref{KKT1} must hold, one can easily conclude that \(y = 0\). However, minimizing (or maximizing) the null function is not interesting, because every point that respects the constraints is a solution. I can ignore this case.
        \item \(x^TQx = 1\)\par
        Let's find the value of \(x\), given that \eqref{KKT1} must hold:
        \[y^T=2\mu x^TQ\]
        \[y^TQ^{-1}=2\mu x^T\]
        \begin{equation}\label{xT}
            x^T = \frac{y^TQ^{-1}}{2\mu}
        \end{equation}
        Moreover, since \(Q^{-1}=(Q^{-1})^T\) (\(Q\) is symmetric and positive definite):
        \begin{equation}\label{x}
            x = \frac{(y^TQ^{-1})^T}{2\mu} = \frac{(Q^{-1})^T(y^T)^T}{2\mu} = \frac{Q^{-1}y}{2\mu}
        \end{equation}
        Now, I can plug the values found at \eqref{xT} and \eqref{x} into \(x^TQx=1\):
        \[x^TQx = \frac{y^TQ^{-1}}{2\mu}Q\frac{Q^{-1}y}{2\mu} = \frac{y^TQ^{-1}QQ^{-1}y}{4\mu^2} = \frac{y^TQ^{-1}y}{4\mu^2} = 1\]
        \[\mu^2=\frac{y^TQ^{-1}y}{4}\]
        \[\mu=\frac{1}{2}\sqrt{y^TQ^{-1}y}\]
        Is this solution an optimal one? Let's compute the Hessian of the Lagrangian function:
        \[\nabla^2L(x,u) = 2\mu Q\]
        You can now notice that
        \[z^T\nabla^2L(x,u)z > 0, z\neq 0\]
        This is due to the fact that \(Q\) is a positive definite matrix. Thus, I can conclude that this is an optimal solution for the problem. Which is the value of the function in this point? I just have to plug into the function of the original problem the solution (\(x\) and \(\mu\)) that I found:
        \[y^Tx = x^Ty = \frac{y^TQ^{-1}}{2\mu}y = \frac{y^TQ^{-1}y}{2\frac{1}{2}\sqrt{y^TQ^{-1}y}} = \sqrt{y^TQ^{-1}y}\]
        
        Now I want to prove that the inequality \((x^Ty)^2 \leq (x^TQx)(y^TQ^{-1}y)\) holds. To show this fact, you just have to use the previous result: the optimal solution is given by \(x = \frac{Q^{-1}y}{2\mu}\) and \(\mu=\frac{1}{2}\sqrt{y^TQ^{-1}y}\), and the related value of the function is \(x^Ty = \sqrt{y^TQ^{-1}y}\).
        \[(x^Ty)^2 \leq (x^TQx)(y^TQ^{-1}y)\]
        \[(\sqrt{y^TQ^{-1}y})^2 \leq (x^TQx)(y^TQ^{-1}y)\]
        \[(y^TQ^{-1}y \leq (x^TQx)(y^TQ^{-1}y)\]
        \[y^TQ^{-1}y \leq (y^TQ^{-1}y)\]
        Clearly the inequality is always true. Notice that the last step is given by the fact that we found the optimal solution setting \(x^TQx = 1\).
    \end{enumerate}
\end{proof}